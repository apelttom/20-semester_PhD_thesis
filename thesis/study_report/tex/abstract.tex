\section*{Abstract}
%\addcontentsline{toc}{chapter}{\protect\numberline{}{Abstract and contributions}}

Humanity is entering a new era of modern, real-time embedded hardware systems that merge together physical world with the world of discrete algorithmic calculations. Such systems are generally referred as Cyber-Physical systems. The design process of such devices is an uneasy task, which is why we need a solid theoretical background for engineers to produce it on a massive scale.

Verification and testing of Cyber-Physical systems play a crucial role in the development framework and researchers all over the world face many challenges that it possesses. This \thesis{} focus on the area of automated testing in the context of modelling Cyber-Physical systems using conventional tools like MATLAB/Simulink.

We present obstacles that arise when verifying such complex systems in an unpredictable environment and possible solutions that have to be experimentally tested and measured. We also propose a different approach to the verification by utilizing as much information about the model as we can.

\bigskip

\noindent{\bf Keywords:}

~cyber-physical systems, model-based development, metric temporal logic, model based testing, Simulink, real-time systems, safety-critical systems, embedded
computers, simulation of models, automotive

\vfill

\section*{Acknowledgement}
This research has been partially supported 
by the Grant Agency of the Czech Technical University in Prague, grant No. SGS17/213/OHK3/3T/18.