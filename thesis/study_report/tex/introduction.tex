\chapter{Introduction}
\enlargethispage*{5pt}

At the beginning of the 21.st century human race enters into a new era of industrial revolution generally called Industry 4.0[TODO citation]. Until now humans used computers and automation to make industrial processes as efficient as possible. But now technology allowed us to create Cyber-Physical Systems (CPS) and integrate them into the industrial process. All the work can be handeled by fully autonomous devices that man will only oversee thus giving us more space for something humans do the best, intellectual creativity. But if we are to put all the work on CPS, we must make sure that such devices will be as safe and secure as possible.

\section{Motivation}
Especially nowadays when creation of CPS is still very expensive, we desperately need an efficent metodology for such task. Many companies all over the world use Model-based design (MBD) for prototyping and enhancing their products. MBD puts narurally a lot of emphasis on the creation of digital model of CPS. An important part of such process is model verification. Usually an engineer has a list of requirements that CPS must comply in order to be considered as safe an secure. Manual process of verification of models of CPS is very time consuming and limited. Several verification tools have been developed to address this issue by running automated tests against a set of requirements in a Simulation. These tools usually use complex search algorithms to find a simulation trace that violates given requirement(s). It is not a trivial task and in addition such tools treat models only as black box, not considering its inner structure. This approach thus have its limitations and that is why we propose new algorithms for automated testing of models of CPS with consideration of their inner structure.


\section{Problem Statement}
Brief description of the topic of the \thesis. A complete explanation of the topic shall be described within chapter \ref{chap.stateoftheart} at page \pageref{chap.stateoftheart}.

\section{Related Work/Previous Results}
Briefly.

\section{Structure of the \Thesis{}}
The \thesis{} is organized into ... chapters as follows:
\begin{enumerate}
\item \emph{Introduction}: Describes the motivation behind our efforts together with our goals. There is also a list of contributions of this \thesis. 

\item \emph{Background and State-of-the-Art}: Introduces the reader to the necessary theoretical background and surveys the current state-of-the-art.

\item \emph{Overview of Our Approach}: ...

\item \emph{Prelimitary Results}: ...

\item \emph{Conclusions}: Summarizes the results of our research, suggests possible topics of your doctoral thesis and further research, and concludes the \thesis.
\end{enumerate}
