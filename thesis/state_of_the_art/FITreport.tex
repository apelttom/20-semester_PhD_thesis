% Example use of document class for research/technical/research review reports
% at CTU FIT (http://fit.cvut.cz)
% 2015/04/12 Created by Ondrej Guth <ondrej.guth@fit.cvut.cz>



% Nutno napsat v souladu s metodickym pokynem pro vydavani 
% vyzkumnych, souhrnnych a technickych zprav na FIT CVUT




% % % % 
% OPTIONS of the class
% LANGUAGE:
% czech 
% english
% TYPE
% research
% technical
% review
% FONT NORMALSIZE
% 10pt
% 11pt
% 12pt
% % 
\documentclass[english,technical,10pt]{FITreport}[2018/01/26]

\usepackage[utf8]{inputenc}

\usepackage{lmodern} % unicode version of Computer Modern font
\usepackage{amsthm}
% \usepackage{hyperref}

% % % % % 
% Use ALL of the following commands
% % % % % 

\title{Example research report\thanks{The research has been supported by\dots}}
\author{Pavel Tvrd{\' i}k\thanks{The author has been supported by\dots}, Second Author\thanks{The author's research has been supported by\dots}\affil{Department of Computer Systems\\\theFIT}
	\and Third Author\affil{Another University}}
\abstractL{Abstract in language of this report.}
\abstractEN{Abstract in English (used only with reports written in Czech).}
\keywordsL{list of keywords in language of this report}
\keywordsEN{list of keywords in English (used only with reports written in Czech).}
\reportNumber{01}
\reportYear{15}
\published{April 2015}

% % % % % % % % % % 

\begin{document}

\theoremstyle{definition}
\newtheorem{lemma}{Lemma}
\newtheorem{theorem}[lemma]{Theorem}
\newtheorem{definition}[lemma]{Definition}
\newtheorem{preposition}[lemma]{Preposition}
\newtheorem{example}[lemma]{Example}
\newtheorem{corollary}[lemma]{Corollary}
\newtheorem{proposition}[lemma]{Proposition}
\newtheorem{property}[lemma]{Property}
\newtheorem{observation}[lemma]{Observation}
\theoremstyle{remark}
\newtheorem{notation}[lemma]{Notation}
\newtheorem{note}[lemma]{Note}




\section{Introduction}

\subsection{Some Subsection}

\begin{definition}[Some term]
    This is our definition.
\end{definition}

\begin{corollary}
    This is some corollary.
\end{corollary}

You may generate list of references using iso690 Bib\TeX{} style, a non-complete implementation of \cite{iso690}.

\bibliographystyle{iso690}
\bibliography{mybibliographyfile}
    
\end{document}
