\chapter{Introduction}
\enlargethispage*{5pt}

At the beginning of the 21.st century human race enters into a new era of industrial revolution generally called Industry 4.0[TODO citation]. So far humans used computers and automation to make industrial processes as efficient as possible. But now the technology allowes us to create Cyber-Physical Systems (CPS) and integrate them into the industrial process. All the work thus can be passed to fully autonomous devices that man will only oversee, giving us more space for something humans do the best, intellectual creativity. But if we are to put all the work on CPS, we must make sure that such devices will be as safe and secure as possible.

\section{Motivation}
CPS are specific in their strusture[5[Tariq,Florence]Design Specification of Cyber-Physical Systems]. They contain both a discrete unit and continuous unit. Most of such devices fit into cathegory of embedded systems, because they monitor variables of the physical world (temperature, pressure, chemical composition, speed, etc.) and also react based on the values of such variables. The manufacturing process of CPS is still very expensive. To address this issue, many companies all over the world use Model-based design (MBD) for prototyping and upgrading their products. MBD puts a lot of emphasis on the creation of digital model of CPS.

An important part of such process is model verification. Usually an engineer has a list of requirements that CPS must comply in order to be labeled as safe an secure. Manual process of verification of models of CPS is very time consuming and limited. That is why several verification tools have been developed to help companies by running automated tests simulations against a set of requirements. These tools use complex search algorithms to find a simulation trajectory that violates given requirement(s). It is not a trivial task, because of the conjuction of discrete and continuous worlds. For example continuous dynamic of rotating car wheel can be clearly described using set of differential equations, but when an Anti-lock braking system discrete controller locks the wheel, none of these equations holds.

\section{Problem Statement}
In addition to a vast complexity of behaviour of CPS, verification tools treat models of CPS only as black boxes, not considering its inner structure. This approach have its limitations. Testing a model without the knowledge of its inner structure will never be as effective as if we would test it with structure and contextual analysis. That is why we aim to propose new algorithms for automated testing of models of CPS with the consideration of their inner structure.

The first objective of our research is to gather useful models of CPS, preferably the ones used in the industry area. Then we focus on the verification process itself and the tools generally used in practise. We try to find use cases when the performance of conventional or academic tools is insufficient and enhance them by providing deep model analysis information.

\section{Related Work/Previous Results}
This research is based upon the work of the research group from Cyber-Physical Systems Laboratory at Arizona State University. [TODO publication citations] They created a verification tool named S-TaLiRo [TODO citation of S-TaLiRo paper] and also presented their own metric for effective searching for simulation trajectories when verifying a model against given specification [citation of robustness metric paper].

When working with specification and requirements we use Metric Temporal Logic (MTL) developed by Ron Koymans [citation]. This way of specifying demands on CPS is suitable because MTL allows us to formulate restrictions as: "There is a maximum number of time units so that each occurrence of an event E is responded to within this bound".

Our efforst were presented on the student seminar PAD 2017 [citation] where we gathered a lot of valuable feedback. This helped us to concreticise our goals and form reasonable milestones.

\section{Structure of the \Thesis{}}
The \thesis{} is organized into ... chapters as follows:
\begin{enumerate}
\item \emph{Introduction}: Describes the motivation behind our efforts together with our goals. There is also a list of contributions of this \thesis. 

\item \emph{Background and State-of-the-Art}: Introduces the reader to the necessary theoretical background and surveys the current state-of-the-art.

\item \emph{Overview of Our Approach}: ...

\item \emph{Prelimitary Results}: ...

\item \emph{Conclusions}: Summarizes the results of our research, suggests possible topics of your doctoral thesis and further research, and concludes the \thesis.
\end{enumerate}
