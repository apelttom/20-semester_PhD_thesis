\documentclass[conference,a4paper]{IEEEtran-CZ}

\usepackage[utf8]{inputenc}

\usepackage[czech]{babel}


\usepackage{amsmath}
\usepackage{alltt}
\usepackage{verbatim}

\usepackage[pdftex]{graphicx}
\graphicspath{{pictures/}}
\usepackage{subcaption}

\usepackage[usenames,dvipsnames]{color}

\usepackage{cite}

\usepackage{algpseudocode}
\usepackage{algorithm}
% alg
\makeatletter
\renewcommand{\ALG@name}{Algoritmus}
\def\BState{\State\hskip-\ALG@thistlm}
\newcommand{\Break}{\State \textbf{break}}
\makeatother


% Oprava cline v [czech]{babel}
\makeatletter
\begingroup
\toks0=\expandafter{\@cline{#1}-{#2}\@nil}
\@ifpackageloaded{booktabs}{%
  \toks2=\expandafter{\@@@cmidrule[{#1}-{#2}]{#3}{#4}}%
}{}
\catcode`-=\active
\edef\x{\gdef\unexpanded{\@cline#1-#2\@nil}{\the\toks0}}\x
\@ifpackageloaded{booktabs}{%
  \edef\x{\gdef\unexpanded{\@@@cmidrule[#1-#2]#3#4}{\the\toks2}}\x
}{}
\endgroup
\makeatother



%\usepackage{flushend}

\begin{document}

\title{State of the art}

\author{\IEEEauthorblockN{Tomáš Apeltauer}
2. ročník prezenčního studia \\
Školitel: Stefan Ratschan \\
\\
\IEEEauthorblockA{České vysoké učení technické v Praze, Fakulta informačních technologií\\
Thákurova 9, Praha, 16000\\
apelttom@fit.cvut.cz}
}

% make the title area
\maketitle


\begin{abstract}


\end{abstract}

\IEEEpeerreviewmaketitle


\begin{IEEEkeywords}

\end{IEEEkeywords}

\maketitle

\section{Úvod}

Pro systémy, které kombinují fyzikální svět se světem výpočtů, používáme pojem kyber-fyzikální systém\cite{Rajeev:2015}.
Tato těsná interakce fyzikálního světa a světa výpočtů má za následek vyšší složitost 
kyber-fyzikálních systémů, protože spojením obou oblastní dostáváme mnohonásobně 
větší množinu stavů a situací, do kterých se může kyber-fyzikální systém dostat a 
obsáhnout je všechny není možné.

\subsection{Problematika testování kyber-fyzikálních systémů}

Navzdory významnému pokroku v rámci technologie kyber-fyzikálních systémů, stále postrádáme
dostatečně vyspělý výzkum, který by zaštítil oblast vysoce spolehlivých kyber-fyzikální systémů.
Důsledkem toho nezvládají současné analytické nástroje plně pokrýt komplexitu kyber-fyzikálních systémů,
nebo adekvátně predikovat jejich chování. Příkladem je Internet věcí, který se neustále rozmáhá, 
a jenž má potenciál škálovat do úrovní biliónů propojených zařízení, která dokáží monitorovat, 
kontrolovat i jinak interagovat s fyzickým prostředím okolo nás. Přirozeně jsou zde kladeny 
vysoké nároky na spolehlivost, bezpečnost a robustnost takových systémů\cite{NSF:CPS:2016}.

\subsection{Význam oblasti testování kyber-fyzikálních systémů}

V průmyslové sféře je hojně využíván přístup tvorby a aplikace abstraktních modelů 
v procesu návrhu (Model-Based design\cite{Jensen:MBD:2011}).
Model-Based design nám umožňuje simulovat, testovat a verifikovat výsledný systém 
už v raných fázích procesu návrhu. Příkladem nástrojů z praxe muže být software 
\emph{MATLAB/Simulink}, \emph{Statemate}, nebo software \emph{Modelica}, příkladem nástroje z 
akademického prostředí je \emph{Ptolemy} (UC Berkeley).

Softwarové komponenty proto již nejsou výhradně psány pouze v C, nebo Assembleru,
ale stále častěji modelovány pomocí výše zmíněných nástrojů a tak nabývá na významu
i oblast testování těchto modelů\cite{Bringmann:MBT:2008}. Současně existuje 
velmi silná motivace proces testování automatizovat a snížit tak náklady na vývoj modelů 
kyber-fyzikálních systémů. Navíc bychom tím dokázali zvýšit použitelnost již vytvořených 
testovacích scénářů.

\subsection{Obtížnost problematiky testování}

Hlavním zdrojem obtížnosti v oblasti testování modelů kyber-fyzikálních systémů je složitost nástrojů,
velký počet různých toolboxů a absence jasně definované, standardizované formální sémantiky v 
programech jako je například Simulink. Modely v tomto nástroji se sestávají z funkčních bloků
a každý z nich má jasně definované vstupní i výstupní kanály. Tyto bloky nefungují
izolovaně ale mohou si předávat data pomocí námi určených komunikačních rozhranní,
navíc je možné modely hierarchicky strukturovat, protože jeden funkční blok lze reprezentovat
i jako množinu podbloků a jejich rozhranní. Takto lze v programu Simulink vytvářet složité
komplexní modely, které reálně reprezentují v praxi využívané systémy.

\section{Vymezení oblasti výzkumu}

Pro modely vytvořené softwarem Simulink zatím existují pouze black box algoritmy, nebo
toolboxy s omezenou funkcionalitou, například \emph{T-Vec Tester} a 
\emph{Reactive Systems Reactis Tester}\cite{Blackburn:1996,Sims:2007}.
Tyto toolboxy pracují s funkčními bloky Simulinku bez nutné znalosti vnitřní hierarchické struktury,
staví zejména na definovaném komunikačním rozhranní a dále pak na jasně formulované
specifikaci systému. Navíc pro svou optimalizaci využívají algoritmy black-box 
optimalizace\cite{Gendreau:2010}. 

\subsection{Omezení}

Tyto nástroje mají své praktické využití při verifikaci
konzistence modelů z pohledu validní manipulace s daty (dělení nulou, přetečení), nebo
při kontrole metrik jako jsou \emph{state coverage}, \emph{branch coverage} a 
hlavně \emph{MD/DC coverage}, ale pro otestování modelů pod intenzivní zátěží, v situacích 
simulujících pokud možno co nejvěrohodněji reálné případy užití, jsou tyto nástroje 
nedostačující. Dokáží generovat takové testy, aby dosáhli vysoké míry pokrytí kódu a
jsou schopné pracovat i se stavovými diagramy. Bohužel neuvažují vnitřní strukturu 
modelů, použité materiály, fyzikální veličiny a zákony, což se v praxi může projevit 
selháním systému za určitých specifických okolností. Navíc jsou omezeny velikostí
generovaných testů (potažmo délkou generovaných signálů) a zvládnou zpracovat
modely jen do určité míry složitosti.

\subsection{White-box testing}

Algoritmy, které využívají vnitřní strukturu modelů a jsou určeny pro testování a verifikaci,
zatím existují pouze pro hybridní dynamické systémy\cite{Dang:2009,Plaku:2013,Zutshi:2013,Ratschan:HS:2014}.
Model hybridního dynamického systému vznikl v akademické obci právě pro účely 
automatického testování modelů\cite{Schaft:HDS:2000}. Modely jsou jednodušší než 
ty vytvořené v Simulinku, ale mají jasnou sémantiku.

\section{Cíle práce}

Cílem mé disertační práce je vývoj algoritmů pro automatické testování kyber-fyzikálních systémů
nad modely softwarových nástrojů, běžně používaných v praxi. Algoritmy by měly využívat 
vnitřní strukturu modelu, ale zároveň by měly fungovat v případě, že rozšíříme modely o prvky,
které se běžně vyskytují v průmyslové praxi. Věnuji se snaze o aplikování obecných algoritmů 
pro testování a verifikaci hybridních dynamických systémů na modely, jenž nemají jasně definovanou 
formální sémantiku.

\section{Hybridní systémy}

Hybridní dynamické modely používáme pro modelování hybridních systémů, které obsahují 
jak spojitou část, jejíž vývoj závisí na čase, tak diskrétní část. Hybridní dynamické modely 
nám umožňují lépe pracovat se spojitým světem a změnami závislými na čase, pomocí 
diferenciálních a algebraických rovnic. Tato vlastnost naneštěstí komplikuje automatizaci testování 
a verifikaci abstraktních modelů.

Hybridní dynamický model můžeme reprezentovat pomocí hybridního stavového 
automatu, který vychází z klasického stavového automatu. Pro zobrazení používáme 
stavový diagram, jak je ukázáno na obrázku \ref{fig:bouncingBall}. Diskrétní část systému 
je zobrazena pomocí stavů a přechodů mezi nimi. Je definován počáteční stav. 
Přechody jsou definovány pomocí přechodových podmínek, ve kterých mohou figurovat 
předem definované konstanty. Dále se tu objevují proměnné typu \texttt{cont}, které 
nabývají hodnot z množiny reálných čísel (nebo intervalu reálných čísel) 
a jsou aktualizovány spojitě spolu s ubíhajícím časem, zatímco proces čeká v určitém stavu.

\begin{figure}
\centering
\includegraphics[scale=0.41]{hybridModel}
\caption{Hybridní model odrážejícího se míče}
\label{fig:bouncingBall}
\end{figure}

\subsection{Hybridní proces}

Hybridní dynamické modely se úzce pojí s pojmem hybridního procesu, a během hybridního procesu, 
který nám poskytuje nástroj pro vyjádření takového modelu v čase. Hybridní proces se skládá z:\cite{Rajeev:2015}

\begin{enumerate}
  \item Asynchronního procesu \textit{P}, kde jsou některé vstupní, výstupní a stavové proměnné typu \texttt{cont}
  \item Časově-spojitého invariantu \textit{CI}, reprezentovaného booleovským výrazem nad stavovou proměnnou \textit{S}
  \item Pro každou vstupní proměnnou $y$ typu \texttt{cont}, výrazem ohodnocení $h_y$ nad stavovými a vstupními proměnnými typu \texttt{cont}
  \item Pro každou stavovou proměnnou $x$ typu \texttt{cont}, výrazem ohodnocení $f_x$ nad stavovými a vstupními proměnnými typu \texttt{cont}
\end{enumerate}

\medskip

Dále platí, že vstupy, výstupy, stavy, počáteční stavy, vnitřní děje, vstupní děje a výstupní děje 
hybridního procesu \textit{HP} jsou stejné, jako u asynchronního procesu \textit{P}. Pro daný stav $s$, časový úsek 
$\delta > 0$ a vstupní signál $\bar{u}$ pro každou vstupní proměnnou $u$ typu \texttt{cont} na intervalu
$[0,\delta]$ je odpovídajícím časovým dějem procesu \textit{HP} diferenciovatelný stavový signál  $\bar{S}$
nad stavovými proměnnými a signál  $\bar{y}$ pro každou vstupní proměnnou $y$ typu \texttt{cont} nad
intervalem $[0,\delta]$ takový, že:\cite{Rajeev:2015}

\begin{enumerate}
  \item Pro každou stavovou proměnnou $x, \bar{x}(0) = s(x)$
  \item Pro každou diskrétní stavovou proměnnou $x$ a čas $0 \leq t \leq \delta, \bar{x}(t) = s(x)$
  \item Pro každou výstupní proměnnou $y$ typu \texttt{cont} a čas $0 \leq t \leq \delta$ se $\bar{y}(t)$ rovná $h_y$ vyhodnoceného pomocí
	hodnot $\bar{u}(t)$ a $\bar{S}(t)$
  \item Pro každou každou stavovou proměnnou $x$ typu \texttt{cont} a čas $0 \leq t \leq \delta$ se derivát času $(d/dt)\bar{x}(t)$ rovná 
	$f_x$ vyhodnocené na základě hodnot $\bar{u}(t)$ a $\bar{S}(t)$
  \item Pro všechna $0 \leq t \leq \delta$ splňují časově-spojitý invariant $CI$ hodnoty $\bar{S}(t)$ nad stavovými proměnnými v čase $t$
\end{enumerate}

\subsection{Zenónův běh hybridního procesu}

Nekonečný běh hybridního procesu \textit{HP} se nazývá \textit{Zenónovým během}, pokud je suma 
časových úseků všech měřených dějů v daném běhu ohraničena konstantou. Stav $s$, náležící
hybridnímu procesu \textit{HP} se nazývá \textit{Zenónovým stavem}, pokud každý konečný běh, 
který obsahuje stav $s$ je Zenónovým během. Hybridní proces \textit{HP} se nazývá 
\textit{Zenónovým procesem}, pokud obsahuje stav $s$, který je dosažitelný a zároveň je Zenónovým 
stavem.

Pokud vyvozujeme závěry za pomoci Zenónových běhů hybridních procesů, nedospějeme ke korektním závěrům. 
Přítomnost  jediného Zenónova procesu může mít nepředvídatelný vliv na analýzu celého systému, proto bychom
se měli Zenónovým komponentám  během formálního modelování vyhnout. Zenónův proces lze převést
do tvaru, který nevyhnutelně nevyžaduje přepínání stavů po stále kratší a kratší době, čímž lze Zenónovu
vlastnost odstranit.\cite{Rajeev:2015}

\subsection{Stabilita hybridních systémů}

Další z důležitých vlastností hybridních procesů je jejich stabilita. Vzhledem k faktu, že hybridní procesy
obsahují přepínání stavů, není možné použít v tomto případě matematickou analýzu, užívanou pro 
charakteristiku stability lineárních systémů, ani přidružené techniky pro návrh stabilizačních kontrol.
Analýza stability hybridních systémů je velmi náročný problém. Využívání analyzačních technik z teorie 
spojitých systémů na hybridní systémy zůstává aktivní oblastí vědeckého výzkumu.\cite{Rajeev:2015}

\section{Simulink}

Modelování kyber-fyzikálních systémů není otázka čistě akademická, ale je často využíváno i v praxi.
Nejrozšířenějším nástrojem pro Model-based design v průmyslu je software Simulink. Softwarový 
nástroj nabízí obsáhlou knihovnu komponent s jejíž pomocí jsme schopni systém popsat, 
většinou skrze matematické rovnice a algebraické operace.

\subsection{Metodika modelování}

Proces modelování je rozdělen do několika fází:

\begin{enumerate}
  \item Stanovení cílů a požadavků na model (jaké otázky nám zodpoví, požadavky na přesnost, definice problému)
  \item Vymezení systémových komponent (identifikování fyzikální a kybernetické části modelu, vztahy mezi komponentami)
  \item Definice rovnic popisujících systém (v případě kyber-fyzikálních systémů se často jedná o diferenciální rovnice)
  \item Tvorba sady parametrů (seznam konstant, koeficientů a jejich hodnot - získané např. měřením)
  \item Proces tvorby modelu (v software Simulink pomocí grafické reprezentace)
 	\begin{itemize}
	  \item Vytvoření bloku pro danou komponentu
  	  \item Validace komponenty pomocí simulace chování komponenty
	\end{itemize}
  \item Integrace komponent mezi sebou a validace jejich vzájemné spolupráce (využití simulace)
\end{enumerate}

Proces validace komponenty pomocí simulace, případně pomocí kontroly formálních požadavků
již z velké části pokrývá balíček \emph{Simulink Verification \& Validation Toolbox}. Ten dokáže automaticky 
kontrolovat požadavky kladené na komponenty, validovat oproti průmyslovým standardům 
(\emph{ISO 26262}, \emph{DO-1788}) a kontrolovat shodu oproti formálnímu popisu. Nepracuje
však se všemi prvky, které jsou v praxi nezbytné, což otevírá dveře dalšímu výzkumu.

\subsection{Integrační testování}

Otázka integračního testování v nástroji Simulink je velmi komplexní a zahrnuje užívání formálních metod, jako jsou
MC/DC pokrytí, nebo automatické generování testů. I přes velkou snahu nedosahují současné
techniky požadované kvality a míry pokrytí proměnných.

\section{Závěr}

Článek nastínil problematiku složitosti automatického testování modelů kyber-fyzikálních systémů
a představil oblasti, které z akademického hlediska nabízí zajímavé a jen částečně 
probádané problémy.  Zmíněna byla i motivace, která stojí za úsilím 
objevit a zformulovat praktiky pro automatické testování a případný dopad do praxe.

V druhé části článku je představen pojem hybridního dynamického modelu, s ním spojený
pojem hybridního procesu, vlastnosti Zenónova běhu procesu a nakonec i netriviální otázka, 
jenž se týká stability hybridních systémů. Dále je popsán konkrétní softwarový nástroj Simulink,
hojně využívaný zejména v průmyslové sféře. Byly identifikovány možnosti testování,
které nástroj sám nabízí. Zmíněny byly také omezení nástroje Verification \& Validation Toolbox 
a otevřené otázky, jenž by si zasloužily hlubší analýzu.

\subsection{Další směřování výzkumu}

Výzkum bude nadále mapovat oblast nástrojů používaných pro Model-Based developement, 
jakými jsou např. TALIRO-TOOLS, Statemate, MATRIXX, LabVIEW, JModelica.org, 
nebo Ptolemy a zároveň bude hledat nové způsoby využití již existujících algoritmů 
pro testování hybridních dynamických systémů v problematice testování modelů bez jasné formální
sémantiky. Součástí výzkumu budou otázky detekce Zenónových běhů, jejich transformace
a využití analytických technik z teorie spojitých systémů pro stabilitu hybridních systémů.

\section*{Poděkování}

Tento výzkum byl částečně podporován z projektu ČVUT
\texttt{SGS17/213/OHK3/3T/18}.

\bibliographystyle{IEEEtran}
\def\refname{Literatura}
\bibliography{references}

\vspace{0cm}

\end{document}
