\chapter{Prelimitary Results}
\label{chap.mainresults}
% If you have a lot of preliminary results, feel free to split this chapter into two or more chapters.

Here we will present experiments using the tools and approach we discussed in previous chapters. We are using S-TaLiRo "A tool for temporal logic falsification" to test acquired models of CPS. The testing is based upon falsification process of properties that define or restrict the behavior of given CPS. All properties are written in MTL so a model of CPS can be evaluated in time using MATLAB/Simulink simulation process.

We aim to discover cases when the performance of S-TaLiRo is insufficient for practical use, and thus this tool should be replaced/enhanced in order to handle falsification for a wide range of models of CPS. 

\section{Air-Fuel Ratio Control System}

The first system that we used for our experiments was a model of Air-Fuel Ratio Control System. The model represents a fuel control system for a gasoline engine. The system is highly robust in that individual sensor failures are detected, and the control system is dynamically reconfigured for uninterrupted operation. Traditional signal flow is handled in Simulink while changes in control configuration are implemented in Stateflow.

The air-fuel ratio is computed by dividing the air mass flow rate (pumped from the intake manifold) by the fuel mass flow rate (injected at the valves). The ideal (i.e., stoichiometric) mixture ratio provides a good compromise between power, fuel economy, and emissions. The target air-fuel ratio for this system is 14.6. Typically, a sensor determines the amount of residual oxygen present in the exhaust gas. This gives a good indication of the mixture ratio and provides a feedback measurement for closed-loop control. If the sensor indicates a high oxygen level, the control law increases the fuel rate. When the sensor detects a fuel-rich mixture, corresponding to a very low level of residual oxygen, the controller decreases the fuel rate.

There are four fault switches for simulation of a sensor fault and also a throttle command input. If there is 1 on the switch input, then sensor readings equal to zero, or in case of exhaust gas to 12. We have connected S-TaLiRo to all switches and throttle input and let it watch over air/fuel ratio by sending it as an output from the model to the MATLAB workspace. We defined a simple MTL requirement: "Air/Fuel ratio has to stay under 20". There are no initial conditions, and we restricted the input signal of the throttle from 10 to 20, and it will have 24 control points. We also changed the interpolation function for the signal to linear interpolation in order to mimic the original input throttle signal as much as possible. All four switches range from 0 to 1. Each simulation run will take 50-time samples. For Formally it is written as:

\lstinputlisting[frame=single]{matlab/sTaLiRo_fuel_rate_control.m}

S-TaLiRo was able to falsify given requirement by the initial sample with initial robustness value -214.7867 \cite{Fainekos:RobustnessFinite}. It ran for 17.8684 seconds. We have used variable-step simulation with default ode45 solver.

It is clear that such trivial MTL restriction is too simple for S-TaLiRo. We will continue experimenting with more complex requirements, and we will explore S-TaLiRo's performance on other models of CPS (e.g., the EV powertrain model, EMBS model, Missile Guidance System model, etc.).
 
